
\section*{Abstract}: 

last draft: 
Las redes neuronales son revolucionarias, hoy en día esto es evidente.
Como lo demostró el Teorema Universal de Aproximación, una red neuronal con suficiente cómputo puede aproximar cualquier función, y en este mundo TODA acción, fenómeno o conocimiento se puede representar con una función. Eso quiere decir que las redes neuronales pueden potencialmente hacerlo todo.
Hace más de 30 años, esta realización llevó a los primeros partidarios de las redes neuronales -hoy los padres de la Inteligencia Artificial- a apostarlo todo por lo que hoy se conoce como el deep learning. Esta apuesta dio lugar a los Generative Pre-trained Transformers, o modelos GPT, que hoy lideran la revolución de la IA. ChatGPT se lo demostró a todo el mundo y sentó un antes y un después en la historia de la IA. Esto sin duda es una revolución, pero una que recién empieza. Ahora el conocimiento de todo el mundo está al alcance de la mano de cualquiera, a solo una pregunta de averiguarlo todo. Este parece un momento pivotal para la humanidad, comparable con el invento de la imprenta que permitió que los libros -y por lo tanto el conocimiento- estén al alcance de todos.
Pero por el momento la mejor tecnología GPT es de código cerrado, bajo el argumento de que esta tecnología es muy peligrosa para que el ciudadano común acceda a ella. Es este argumento nuevo? acaso no se argumentó lo mismo en el siglo 15 con la imprenta y los libros de acceso masivo? qué hubiera pasado si un monopolio mundial se hubiera quedado con esta antigua tecnología? qué hace que una tecnología sea benéfica para toda la sociedad y cuál es el rol de la comunidad de código abierto en todo esto?
Esta es una charla sobre cómo un grupo de amigos desarrolladores vieron todo lo que estaba pasando, decidieron meterse y crearon a ALI, el primer Asistente Legal Inteligente que busca democratizar el acceso al conocimiento legal.
Pero sobre todo, esta charla es sobre cómo podés involucrarte con la IA y crear tu propio sistema inteligente.







\section*{Overview}: 
Esta charla explora el advenimiento de la IA de uso masivo, cómo un grupo de amigos desarrolladores se reunieron para desarrollar a ALI -el primer Asistente Legal Inteligente-, y lo más importante de todo, cómo el trabajo de la comunidad de código abierto da lugar a que cualquiera haga lo mismo

Esta charla explora el advenimiento de la inteligencia artificial de uso masivo, el rol de la comunidad de código abierto en esta revolución, cómo un grupo de amigos desarrolladores vieron todo el potencial  y se pusieron a ver qué podían construir. 







\section*{Cuestiones a notar}: 
\begin{itemize}
    \item GPT no implica OpenAI. 
\end{itemize}



\section*{Open questions}:
\begin{itemize}
    \item qué hace que una tecnología impacte benéficamente en la sociedad?
    \item deberíamos regular la IA? a quién beneficia más la regulación?
    \item qué rol juega la comunidad de código abierto en todo esto?
\end{itemize}

